% !TEX root =  ../thesis.tex

This thesis is the result of a part of the research activity that has been conducted for the ``Crowd Sourced Formal Verification'' project, funded by the Defense Advanced Research Projects Agency (DARPA) and carried out by University of Illinois at Chicago, Bell Labs, University of California, Los Angeles and other research groups, In which I was involved as a research assistant.

The project aims at developing a system for proving formal correctness of computer programs. The system would be capable of enhancing automated verifier tools capabilities with human intellectual skills at solving problems, (e.g., puzzle games). This project takes a radically different approach to producing correct software and offers interesting new challenges to formal methods and compiler theory. In particular, the way the users will interact with the program is by playing a puzzle level of a video game. Each level is a translation of the properties of the program to be verified. The result of the crowd-sourced verification will be a set of formally verified properties about a program or a piece of code that an automated verifier alone could hardly infer or prove. The annotations that are sought ensure correctness over critical parts from a security point of view, such as proving buffer overflows or integer overflows. A list of the most common security holes can be found in the Mitre 25 list drawn by CWE\footnote{http://cwe.mitre.org/top25/}. These annotations are then inserted within the program and used by a compiler trained to handle this additional information and exploit it for eventually optimizing the execution of the program and to ensure its correctness.

My personal research activity falls into the project as part of the research to build a compiler, such as the one described above. CSFV is a 3-year long-term project, and the final results are yet to be attained.
